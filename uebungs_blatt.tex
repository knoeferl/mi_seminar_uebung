\documentclass{article}

% Language setting
% Replace `english' with e.g. `spanish' to change the document language
\usepackage[german]{babel}

% Set page size and margins
% Replace `letterpaper' with`a4paper' for UK/EU standard size
\usepackage[letterpaper,top=2cm,bottom=2cm,left=3cm,right=3cm,marginparwidth=1.75cm]{geometry}

% Useful packages
\usepackage{amsmath}
\usepackage{graphicx}
\usepackage[colorlinks=true, allcolors=blue]{hyperref}

\title{Übungen Aktivitätserkennung}
\author{Michael Knöferl \\E-Mail: \href{mailto:Michael.Knoeferl@sutdent.fhws.de}{Michael.Knoeferl@sutdent.fhws.de}} 

\begin{document}
\maketitle

\section{Installation}

Für diese Übung sollte keine Installation auf eurem System notwendig sein.\\ 
Ihr könnte Sie über Google Colab ausführen. \\Hierzu wird jedoch ein Google Konto benötigt.\\
Falls ihr dies nicht habt und keines anlegen wollt könntet ihr die Übung auch auf einem Linux System durchführen.\\
Da hier aber dann mehrere Schritte nötig sind, führe ich dies hier nicht auf. Ihr könnt euch dann gerne an mich wenden und ich helfe euch bei der Installation.\\ \\ 
Die Daten liegen auf Github:
\\ \href{https://github.com/knoeferl/mi_seminar_uebung}{https://github.com/knoeferl/mi\_seminar\_uebung}\newline
\\ 
\\Öffnet die Datei:
\\ \href{https://github.com/knoeferl/mi_seminar_uebung}{uebung\_action\_regognition.ipynb}\newline
\newline
\\Hier findet ihr einen Link der das Dokument in Google Colab öffnet.\newline
 \href{https://colab.research.google.com/github/knoeferl/mi_seminar_uebung/blob/main/uebung_Aktivtaetserkennung.ipynb}{https://colab.research.google.com/github/knoeferl/mi\_seminar\_uebung/blob/main/uebung\_Aktivtaetserkennung.ipynb}
\\
\\Ihr könnt euch eine Instanz mit GPU holen und die Übung durchführen.
\\Beim ersten ausführen kommt eine Warnung, dass das Dokument auf die Google Drive Daten zugreifen kann.
\\ 
\\Links könnt ihr auf ein Inhaltverzeichnis zugreifen um zu den einzelnen Übungen zu gelangen.
\\ 
\\Die Stellen an denen ihr Code einfügen sollt sind mit Kommentaren gekennzeichnet.

\newpage
\section{Übungen}

\subsection{Schreibe ein LSTM}

Im durch Kommentare gekenzeichneten Bereich schreibe ein Aufruf für die LSTM Funktion von Pytorch. \\ \\
Doku LSTM:  \href{https://pytorch.org/docs/stable/generated/torch.nn.LSTM.html}{https://pytorch.org/docs/stable/generated/torch.nn.LSTM.html}\\ \\
Hier könnt ihr mit der Anzahl der Hiddenlayer Spielen und sehen wie sich die Resultate verändern.\\
Die Anzahl der Hiddenlayer muss bei der Linear Funktion als erster Parameter mit hineingeben werden.\\ \\
Doku LINEAR: \href{https://pytorch.org/docs/stable/generated/torch.nn.Linear.html}{https://pytorch.org/docs/stable/generated/torch.nn.Linear.html}

\subsection{LSTM laufen lassen}
Hier könnt ihr die Anzahl der Epochen verändern und in den nachfolgenden Cellen betrachten wie sich die Resultate verändern.

\subsection{Features berechnen} 
\subsubsection{schreibe die Funktion calc\_angle welcher die Winkel veränderung zwischen zwei Bildern zurück gibt} 
\subsubsection{schreibe die Funktion calc\_euclidan welche den Euklisichen Abstand zwischen zwei Punkten zurück gibt} 

\subsection{Ändere den reularization Parameter und betrachte wie sich die Resultate verändern}

Doku SVM: \href{https://scikit-learn.org/stable/modules/generated/sklearn.svm.SVC.html}{https://scikit-learn.org/stable/modules/generated/sklearn.svm.SVC.html}

\end{document}